\documentclass{article}
\usepackage{amsmath, amssymb, amsthm}
\usepackage[utf8]{inputenc}
\newtheorem{theorem}{Theorem}
\title{0}
\author{ProofWiki}
\date{}
\begin{document}
\maketitle

\noindent\textbf{Categories:} categories, subfactorials/examples, fibonacci numbers/examples, square numbers/examples, triangular numbers/examples

\begin{theorem}
The Babylonians from the $2$nd century BCE used anumber basesystem of arithmetic, with a placeholder to indicate that a particular place within a number was empty, but its use was inconsistent.  However, they had no actual recognition ofzeroas a mathematical concept in its own right. The Ancient Greeks had no conception ofzeroas a number. The concept ofzerowas invented by the mathematicians of India. TheBakhshali Manuscriptfrom the $3$rd century CE contains the first reference to it.
\end{theorem}

\begin{proof}
The proof is left as an exercise to the reader.
\end{proof}
\end{document}