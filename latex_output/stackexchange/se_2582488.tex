\documentclass{article}
\usepackage{amsmath, amssymb, amsthm}
\usepackage[utf8]{inputenc}
\title{Will assuming the existence of a solution ever lead to a contradiction?}
\author{Stack Exchange}
\date{}
\begin{document}
\maketitle

\noindent\textbf{Tags:} logic, proof-writing

\section*{Question}
I'm reading Manfredo Do Carmo's differential geometry book. In section 1-7, he discusses the "Isoperimetric Inequality" which is related to the question of what 2-dimensional shape maximizes the enclosed area for a closed curve of constant length. He mentions that A satisfactory proof of the fact that the circle is a solution to the isoperimetric problem took, however, a long time to appear. The main reason seems to be that the earliest proofs assumed that a solution should exist. It was only in 1870 that K. Weierstrass pointed out that many similar questions did not have solutions. This line of reasoning would suggest that assuming the existence of a solution might lead to a contradiction (such as an apparent solution that is not in fact valid). Is this actually a problem? Are there any problems that produce invalid solutions under the (flawed) assumption that a solution exists at all? If so, what is an example and how does it differ from the statement of the isoperimetric problem?

\section*{Answer}
Just the first thing that came to my mind... assume $A=\sum_{n=0}^{\infty}2^n $ exists, it is very easy to find $A $: note $A=1+2\sum_{n=0}^{\infty}2^n =1+2A $, so $A=-1$. Of course, this is all wrong precisely because $A $ does not exist.

\vspace{1em}
\noindent\textit{Score: 71}
\end{document}