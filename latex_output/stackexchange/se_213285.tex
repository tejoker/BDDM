\documentclass{article}
\usepackage{amsmath, amssymb, amsthm}
\usepackage[utf8]{inputenc}
\title{Prove: If a sequence converges, then every subsequence converges to the same limit.}
\author{Stack Exchange}
\date{}
\begin{document}
\maketitle

\noindent\textbf{Tags:} real-analysis, limits, proof-writing

\section*{Question}
I need some help understanding this proof: Prove: If a sequence converges, then every subsequence converges to the same limit. Proof: Let $s_{n_k}$ denote a subsequence of $s_n$. Note that $n_k \geq k$ for all $k$. This easy to prove by induction: in fact, $n_1 \geq 1$ and $n_k \geq k$ implies $n_{k+1} > n_k \geq k$ and hence $n_{k+1} \geq k+1$. Let $\lim s_n = s$ and let $\epsilon > 0$. There exists $N$ so that $n>N$ implies $|s_n - s|  N \implies n_k > N \implies |s_{n_k} - s| < \epsilon$. Therefore: $\lim_{k \to \infty} s_{n_k} = s$. What is the intuition that each subsequence will converge to the same limit I do not understand the induction that claims $n_k \geq k$

\section*{Answer}
A sequence converges to a limit $L$ provided that, eventually, the entire tail of the sequence is very close to $L$. If you restrict your view to a subset of that tail, it will also be very close to $L$. An example might help. Suppose your subsequence is to take every other index: $n_1 = 2$, $n_2 = 4$, etc. In general, $n_k = 2k$. Notice $n_k \geq k$, since each step forward in the sequence makes $n_k$ increase by $2$, but $k$ increases only by $1$. The same will be true for other kinds of subsequences (i.e. $n_k$ increases by at least $1$, while $k$ increases by exactly $1$).

\vspace{1em}
\noindent\textit{Score: 78}
\end{document}