\documentclass{article}
\usepackage{amsmath, amssymb, amsthm}
\usepackage[utf8]{inputenc}
\title{Why are mathematical proofs that rely on computers controversial?}
\author{Stack Exchange}
\date{}
\begin{document}
\maketitle

\noindent\textbf{Tags:} soft-question, proof-writing

\section*{Question}
There are many theorems in mathematics that have been proved with the assistance of computers, take the famous four color theorem for example. Such proofs are often controversial among some mathematicians. Why is it so? I my opinion, shifting from manual proofs to computer-assisted proofs is a giant leap forward for mathematics. Other fields of science rely on it heavily. Physics experiments are simulated in computers. Chemical reactions are simulated in supercomputers. Even evolution can be simulated in an advanced enough computer. All of this can help us understand these phenomena better. But why are mathematicians so reluctant?

\section*{Answer}
What is mathematics? One answer is that mathematics is a collection of definitions, theorems, and proofs of them. But the more realistic answer is that mathematics is what mathematicians do. (And partly, that's a social activity.) Progress in mathematics consists of advancing human understanding of mathematics. What is a proof for? Often we pretend that the reason for a proof is so that we can be sure that the result is true. But actually what mathematicians are looking for is understanding. I encourage everyone to read the article On Proof and Progress in Mathematics by the Fields Medalist William Thurston. He says (on page 2): The rapid advance of computers has helped dramatize this point, because computers and people are very different. For instance, when Appel and Haken completed a proof of the 4-color map theorem using a massive automatic computation, it evoked much controversy. I interpret the controversy as having little to do with doubt people had as to the veracity of the theorem or the correctness of the proof. Rather, it reflected a continuing desire for human understanding of a proof, in addition to knowledge that the theorem is true. On a more everyday level, it is common for people first starting to grapple with computers to make large-scale computations of things they might have done on a smaller scale by hand. They might print out a table of the first 10,000 primes, only to find that their printout isn’t something they really wanted after all. They discover by this kind of experience that what they really want is usually not some collection of “answers”—what they want is understanding. Some people may claim that there is doubt about a proof when it has been proved by a computer, but I think human proofs have more room for error. The real issue is that (long) computer proofs (as opposed to, something simple like checking a numerical value by calculator) are hard to keep in your head. Compare these quotes from Gian-Carlo Rota's Indiscrete Thoughts, where he describes the mathematicians' quest for understanding: “eventually every mathematical problem is proved trivial. The quest for ultimate triviality is characteristic of the mathematical enterprise.” (p.93) “Every mathematical theorem is eventually proved trivial. The mathematician’s ideal of truth is triviality, and the community of mathematicians will not cease its beaver-like work on a newly discovered result until it has shown to everyone’s satisfaction that all difficulties in the early proofs were spurious, and only an analytic triviality is to be found at the end of the road.” (p. 118, in The Phenomenology of Mathematical Truth) Are there definitive proofs? It is an article of faith among mathematicians that after a new theorem is discovered, other simpler proofs of it will be given until a definitive one is found. A cursory inspection of the history of mathematics seems to confirm the mathematician’s faith. The first proof of a great many theorems is needlessly complicated. “Nobody blames a mathematician if the first proof of a new theorem is clumsy”, said Paul Erdős. It takes a long time, from a few decades to centuries, before the facts that are hidden in the first proof are understood, as mathematicians informally say. This gradual bringing out of the significance of a new discovery takes the appearance of a succession of proofs, each one simpler than the preceding. New and simpler versions of a theorem will stop appearing when the facts are finally understood. (p.146, in The Phenomenology of Mathematical Proof). In my opinion, there is nothing wrong with, or doubtful about, a proof that relies on computer. However, such a proof is in the intermediate stage described above, that has not yet been rendered trivial enough to be held in a mathematician's head, and thus the theorem being proved is to be considered still work in progress.

\vspace{1em}
\noindent\textit{Score: 119}
\end{document}