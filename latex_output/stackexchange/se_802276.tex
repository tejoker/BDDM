\documentclass{article}
\usepackage{amsmath, amssymb, amsthm}
\usepackage[utf8]{inputenc}
\title{When working proof exercises from a textbook with no solutions manual, how do you know when your proof is sound/acceptable?}
\author{Stack Exchange}
\date{}
\begin{document}
\maketitle

\noindent\textbf{Tags:} soft-question, proof-writing, self-learning, advice

\section*{Question}
When working proof exercises from a textbook with no solutions manual, how do you know when your proof is sound/acceptable? Often times I "feel" as if I can write a proof to an exercise but most of those times I do not feel confident that the proof that I am thinking of is good enough or even correct at all. I can sort of think a proof in my head, but am not confident this is a correct proof. Any input would be appreciated. Thanks.

\section*{Answer}
Ask a more experienced person. IMHO that's really the only option, and one of the reasons for this is that it is very important for a proof to communicate a result and its justification to another person. If the proof is good enough to convince yourself, that's a start, but the real test is whether you can express it in such a way as to convince someone else. And BTW... the same applies if the textbook does have a solutions manual. Your proof is inevitably going to be different from the one in the book, and it takes a lot of experience and mathematical understanding to decide whether the differences are important or not.

\vspace{1em}
\noindent\textit{Score: 47}
\end{document}