\documentclass{article}
\usepackage{amsmath, amssymb, amsthm}
\usepackage[utf8]{inputenc}
\title{Does notation ever become \textbackslash{}&quot;easier\textbackslash{}&quot;?}
\author{Stack Exchange}
\date{}
\begin{document}
\maketitle

\noindent\textbf{Tags:} notation, proof-writing, self-learning

\section*{Question}
I'm in my first semester of college going for a math major and it's pretty great. I'm doing well, however, there seems to be huge gap between how difficult /complex an idea is and how convoluted it is presented. Let me make an example: In Analysis we discussed the Bolzano Weierstrass theorem and one of the lemmas showed that every sequence in $\mathbb{R}$ has a monotone subsequence. The idea behind the proof with the maximum spots ( speaking colloquially here ) is super simple and pretty elegant if you asked me, but I spent a significant amount of time trying to understand the notation of the professor until I went to this site to read a "proper explanation" of the proof, which had much simpler notation in it. Extracting the idea of the proof took me lots of time because of the strange notation, but once you understand what is going on, it is really easy. Most of the time spent studying lectures is about digging through the formalities. Do I just have to spend more time really going through all the formal details of a proof to become accustomed to that formality? Or do more advanced mathematicians also struggle to extract the ideas from the notation? I'd assume there will come a point, where the idea itself is the most complex part, so I do not want to get stuck at the notation, when that happens. ( Proof Verification - Every sequence in $\Bbb R$ contains a monotone sub-sequence if you are interested )

\section*{Answer}
As others have pointed out, it gets much better if that's your first semester. But in my experience, there is not much relief between, say, years 2 and 4 of your studies. Sure, you get more mature, but the material gets more difficult too. To address your question whether "more advanced mathematicians also struggle to extract the ideas from the notation", I'd like to quote V.I. Arnold, since I think it's exactly in the spirit of your frustration. It is almost impossible for me to read contemporary mathematicians who, instead of saying "Petya washed his hands," write simply: "There is a $t_1<0$ such that the image of $t_1$ under the natural mapping $t_1 \mapsto {\rm Petya}(t_1)$ belongs to the set of dirty hands, and a $t_2$, $t_1<t_2 \leq 0$, such that the image of $t_2$ under the above-mentioned mapping belongs to the complement of the set defined in the preceding sentence.'' The trade-off is clear: without rigor math would've been quite a mess. But if rigor is the only way math gets communicated to someone, this person simply won't have time to get far in math.

\vspace{1em}
\noindent\textit{Score: 100}
\end{document}