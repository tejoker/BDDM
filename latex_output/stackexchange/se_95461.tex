\documentclass{article}
\usepackage{amsmath, amssymb, amsthm}
\usepackage[utf8]{inputenc}
\title{Why do we write proofs \textbackslash{}&quot;forward?\textbackslash{}&quot;}
\author{Stack Exchange}
\date{}
\begin{document}
\maketitle

\noindent\textbf{Tags:} proof-writing

\section*{Question}
I am aware that this might turn into a discussion, but I have a feeling this might have an answer (maybe something historical?) instead. I'm hoping that those with speculations keep it in the comments. I have started to work on formal proof writing this quarter, and I discovered that the key to getting to some of them is to think of the problem "backwards." But, alas, when I wrote my proof starting with this, my professor said I shouldn't do it. But why not? It gives the reader a sense of what motivated this type of proof and allows for more understanding, doesn't it? Mods: Feel free to close, if this turns out to be too much of a discussion. I will be in chat for those willing to discuss this.

\section*{Answer}
One main problem with writing an argument backwards, especially for a student beginning to learn about proofs, is that it would be much more difficult to keep track of what is an assumption and what is a goal. In a proof that $A\implies B$, we should never along the way assume that $B$ is true, otherwise we are being circular; but if the statement of $B$ is written down on your paper already, you might get confused and think you'd already demonstrated it to be true. I'm not saying this will always happen, just that it is a greater risk. While it's true that "thinking backwards" can sometimes be a useful strategy for attacking a problem, and explaining your strategy to the reader can be a good addition to a formal proof, it is not a substitute; one should always be able to explain the argument starting from your given information and axioms, and proceeding to the desired statement completely "forwards". It is essential to get sufficient practice with phrasing your argument this way.

\vspace{1em}
\noindent\textit{Score: 46}
\end{document}