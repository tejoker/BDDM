\documentclass{article}
\usepackage{amsmath, amssymb, amsthm}
\usepackage[utf8]{inputenc}
\title{What really is mathematical rigor? How can I be more rigorous?}
\author{Stack Exchange}
\date{}
\begin{document}
\maketitle

\noindent\textbf{Tags:} proof-writing

\section*{Question}
I'm an undergraduate mathematics student who has received some constructive feedback from two instructors at the end of my exams. Namely, that I am a bit hand-wavey and not always very rigorous. While I greatly appreciate this feedback since I intend to apply to graduate school, I worry because when I hand in assignment, I generally feel the proofs are rigorous already, and I'm not sure what I'm missing. Obviously I will be engaging in a dialogue with my professors about what I can do to be more rigorous, but I'm hoping the experts on this forum can provide some examples or distinctions between rigorous arguments and arguments that are close, but maybe gloss over important details so I can more clearly see the difference. It is worth noting that I have taken a course in mathematical reasoning and logic and did very well in it. Something has happened in the last year or so to reduce the quality of my arguments, or rather, the expectation has gone up and my level of rigor has not matched it. I think another aspect of what makes this a troublesome problem for me is that i typically do really well on homework problem sets. 90% consistently, sometimes with little mistakes. I feel like I am getting mixed signals from some of my classes when I am consistently told I am doing well, but am given this advice. All of this is said without spite or malice, I just sometimes feel confused about my own level of understanding. So if anyone has any general advice, or useful examples of arguments that look rigorous but need to be patched up, that would be greatly appreciated. I want to get a sense for what a really solid proof looks like compared to a less polished argument that looks passable to someone still with naivete in them. Edit: Here is an example of one such question in which I struggled for a long while to produce the proof posted, and even then I was missing something to be fully rigorous. Homology groups of orientable surfaces.

\section*{Answer}
You should be able to delineate the precise mathematical theorems that allow you to make each step in a proof. For example, if you have $(x,y) \in \mathbb{R}^2$ and you write: let $r,\theta$ satisfy $x = r\cos \theta,y=r\sin \theta$ with $r\geq 0$ and $2\pi > \theta \geq 0$, you are using a theorem that says that: Proposition. For all $x,y \in \mathbb{R}$, there exists $r \in \mathbb{R}_{\geq 0}$ and $\theta \in \mathbb{R}$ such that $x=r\cos \theta$ and $y = r \sin \theta$ and $2\pi > \theta \geq 0$. Make sure you can explicitly write out the theorems that allow you to make the steps you are making. The other thing is that you need to develop an intuition for what the instructor (reader, etc.) expects you to take for granted. You may need to write some follow-up lemmas if there are steps in your proof which invoke theorems that you really shouldn't be taking for granted.

\vspace{1em}
\noindent\textit{Score: 47}
\end{document}