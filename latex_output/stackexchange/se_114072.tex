\documentclass{article}
\usepackage{amsmath, amssymb, amsthm}
\usepackage[utf8]{inputenc}
\title{What is the proof that covariance matrices are always semi-definite?}
\author{Stack Exchange}
\date{}
\begin{document}
\maketitle

\noindent\textbf{Tags:} probability, matrices, vector-spaces, proof-writing, positive-semidefinite

\section*{Question}
Suppose that we have two different discreet signal vectors of $N^\text{th}$ dimension, namely $\mathbf{x}[i]$ and $\mathbf{y}[i]$, each one having a total of $M$ set of samples/vectors. $\mathbf{x}[m] = [x_{m,1} \,\,\,\,\, x_{m,2} \,\,\,\,\, x_{m,3} \,\,\,\,\, ... \,\,\,\,\, x_{m,N}]^\text{T}; \,\,\,\,\,\,\, 1 \leq m \leq M$ $\mathbf{y}[m] = [y_{m,1} \,\,\,\,\, y_{m,2} \,\,\,\,\, y_{m,3} \,\,\,\,\, ... \,\,\,\,\, y_{m,N}]^\text{T}; \,\,\,\,\,\,\,\,\, 1 \leq m \leq M$ And, I build up a covariance matrix in-between these signals. $\{C\}_{ij} = E\left\{(\mathbf{x}[i] - \bar{\mathbf{x}}[i])^\text{T}(\mathbf{y}[j] - \bar{\mathbf{y}}[j])\right\}; \,\,\,\,\,\,\,\,\,\,\,\, 1 \leq i,j \leq M $ Where, $E\{\}$ is the "expected value" operator. What is the proof that, for all arbitrary values of $\mathbf{x}$ and $\mathbf{y}$ vector sets, the covariance matrix $C$ is always semi-definite ($C \succeq0$) (i.e.; not negative definte; all of its eigenvalues are non-negative)?

\section*{Answer}
A symmetric matrix $C$ of size $n\times n$ is semi-definite if and only if $u^tCu\geqslant0$ for every $n\times1$ (column) vector $u$, where $u^t$ is the $1\times n$ transposed (line) vector. If $C$ is a covariance matrix in the sense that $C=\mathrm E(XX^t)$ for some $n\times 1$ random vector $X$, then the linearity of the expectation yields that $u^tCu=\mathrm E(Z_u^2)$, where $Z_u=u^tX$ is a real valued random variable, in particular $u^tCu\geqslant0$ for every $u$. If $C=\mathrm E(XY^t)$ for two centered random vectors $X$ and $Y$, then $u^tCu=\mathrm E(Z_uT_u)$ where $Z_u=u^tX$ and $T_u=u^tY$ are two real valued centered random variables. Thus, there is no reason to expect that $u^tCu\geqslant0$ for every $u$ (and, indeed, $Y=-X$ provides a counterexample).

\vspace{1em}
\noindent\textit{Score: 44}
\end{document}