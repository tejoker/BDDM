\documentclass{article}
\usepackage{amsmath, amssymb, amsthm}
\usepackage[utf8]{inputenc}
\title{how to be good at proving?}
\author{Stack Exchange}
\date{}
\begin{document}
\maketitle

\noindent\textbf{Tags:} proof-writing

\section*{Question}
I'm starting my Discrete Math class, and I was taught proving techniques such as proof by contradiction, contrapositive proof, proof by construction, direct proof, equivalence proof etc. I know how the proving system works and I can understand the sample proofs in my text to a sufficient extent. However, whenever I tried proving on my own, I got stuck, with no advancement of ideas in my head. How do you remedy this solution? Should i practise proving as much as possible? So far I've been googling proofs for my homework questions. But the final exam got proving questions (closed-book) so I need to come up with the proofs myself. We mainly focus on proving questions related to number theory. Should I read up on number theory and get acquainted with the properties of integers? I don't know how I should go about becoming proficient in proving. Can you guys share your experience on overcoming such an obstacle? What kind of resources do you use for this purpose? Thank you!

\section*{Answer}
I do not consider myself "good" at proving things. However, I know that I have gotten better. The key to writing a proof is understanding what you are trying to prove, which is harder than it may seem. Know your definitions. Often, I have been hampered or seen students hampered by not really knowing all of the definitions in the problem statement. Work with others. Look at what someone else has done in a proof and ask questions. Ask how they came up with the idea, ask that person to explain the proof to you. Also, do the same for them. Explain your proofs to a classmate and have them ask you questions. Try everything. Students often get stuck on proofs because they try one idea that does not work and give up. I often go through several bad ideas before getting anywhere on a proof. Another good strategy is to work with specific examples until you understand the problem. Plug in numbers and see why the theorem seems to be true. Also, try to construct a counterexample. The reason counterexamples fail often leads to a way to prove the statement.

\vspace{1em}
\noindent\textit{Score: 62}
\end{document}